\chapter{Prehľad problematiky}\label{chap:issues_overview} 

\section{MediaWiki}
MediaWiki je publikačný systém určený na jednoduchú a efektívnu tvorbu obsahu webových stránok založený v roku 1995 Howardom G. Cunninghamom. Tento systém je voľne šíriteľný Open Source softvér, licencovaný pod GNU (General Public Licence), pričom jeho zdrojové kódy sú napísane v jazyku PHP. Na uchovávanie dát využíva relačnú databázu MySQL. Obsah stránky je vytváraný dynamicky, pomocou wikitextu. 

Wikitext je vlastný značkovací jazyk, ktorý zabezpečuje jednoduchú editáciu obsahu webových stránok bez nutnosti ovládať HTML značkovací jazyk a CSS kaskádové štýly. Wikitext je možné písať priamo pomocou spomínanej wikitext syntaxe, kde je možné nadpis sekcie zapísať ako \inlinecode{PHP}{=Nadpis sekcie=}, podnadpis sekcie \inlinecode{PHP}{==Podnadpis sekcie==}. Odkazy na iné stránky sa zapisujú pomocou hranatých zátvoriek \inlinecode{PHP}{[[Názov článku]]}. Podobný zápis sa využíva aj na vkladanie súborových príloh ako napríklad obrázok, kde je syntax zápisu nasledovná: \inlinecode{PHP}{[[File:NazovSuboru.jpg]]}. Ďaľšie informácie ohľadom spôsobu zápisu wikitextu je možné nájsť v dokumentácií MediaWiki systému \citep{MediaWikiHelpFormating}. Obsah stránky je možné zapisovať priamo pomocou tohoto značkovacieho jazyka, alebo pomocou editora formátovaného textu (rich-text editora).

Zmeny vykonané používateľom na obsahu stránky, sú pri ukladaní zapísané do databázy ako samostatná verzia. Históriu zmien (verzií) je možné zobraziť a v prípade potreby taktiež obnoviť. Pri každej verzií je uchovávaný autor danej zmeny, s dátumom kedy zmenu vykonal. 



\newpage
\section{Prehľad technológií}

\subsection{PHP}
PHP (PHP: Hypertext Preprocessor) je populárny open source skriptovací jazyk, ktorý sa používa najmä na programovanie klient-server aplikácií (na strane servera) a pre vývoj dynamických webových stránok.

PHP bolo inšpirované jazykmi podporujúcimi procedurálne programovanie. Najviac vlastností prebralo od jazyka C a jazyka Perl. V neskorších verziách bolo rozšírené o možnosť používať objekty.

Jedna zo zaujímavých vlastností PHP je, že umožňuje oveľa viac ako bežný skriptovací jazyk. Vďaka modulárnemu návrhu možno PHP používať aj na vývoj aplikácii s užívateľským rozhraním (GUI).

PHP dokáže spolupracovať s relačnými databázami, ako napríklad MySQL, Oracle, IBM DB2, Microsoft SQL Server, PostgreSQL alebo SQLite, pričom si stále zachováva jednoduchú a priamočiaru syntax. PHP beží na takmer všetkých najrozšírenejších operačných systémoch, vrátane UNIXu, Linuxu, Windows či Mac OS X. Spolupracuje s najrozšírenejšími webovými servermi.

\subsection{HTML}
Hypertextový značkový jazyk (HyperText Markup Language; HTML) je značkový jazyk určený na vytváranie webových stránok a iných informácií zobraziteľných vo webovom prehliadači. HTML kladie dôraz skôr na prezentáciu informácií (odseky, fonty, váha písma, tabuľky atď.) ako na sémantiku (význam slov).
Pôvodne bol určený ako veľmi zjednodušená podmnožina jazyka SGML, ktorý sa používa v organizáciách s komplexnými publikačnými požiadavkami, ale neskôr sa stal samostatným štandardom (ISO/IEC 15445:2000).
Špecifikáciu jazyka HTML udržiava World Wide Web Consortium (W3C).
Do príchodu HTML5 W3C plánovalo nahradiť HTML jazykom XHTML, implementáciou jazyka XML, ktorá mala zjednodušiť prácu prehliadačov aj tvorcov web stránok.
V mediawiki sa HTML používa na vykreslenie vizuálnych prvkov.

\subsection{Kaskádové štýly (CSS)}
Kaskádové štýly alebo CSS (skratka z angl. Cascading Style Sheets) je všeobecné rozšírenie (X)HTML. Konzorcium W3C[1] označuje CSS ako jednoduchý mechanizmus na vizuálne formátovanie internetových dokumentov.

Štýly umožnili oddeliť štruktúru HTML alebo XHTML od vzhľadu.Prvá verzia CSS (CSS level 1) vznikla už v roku 1996 a umožňovala prácu s písmami, okrajmi a farbami. V roku 1998 bola doplnená o nové možnosti a vznikol CSS level 2. V súčasnosti je podporovaná vo všetkých novších verziách prehliadačov (Internet Explorer, Opera, Mozilla, Netscape, Safari). Aktuálna verzia je CSS level 3.

\subsection{JavaScript}
JavaScript, je skriptovací programovací jazyk. Jazyk je používaný najmä pri tvorbe webových stránok. Pôvodne ho vyvíjal Brendan Eich zo spoločnosti Netscape Communications pod názvom Mocha, neskôr pod menom LiveScript. Pred uvedením na verejnosť bol premenovaný na „JavaScript“, najmä pre vtedajšiu popularitu jazyka Java. Aj na základe jeho názvu je rozšírený názor, že syntax Javascriptu sa podobá Jave, v skutočnosti bol jeho tvorca najviac inšpirovaný jazykom Self.

\subsection{JavaScriptová knižnica Node.js}
Node.js je open-source medziplatformové run-time prostredie pre vykonávanie JavaScript kódu na strane serveru. Historicky bol Javascript určený hlavne pre programovanie na klientskej strane, kde sú skripty písané v Javascripte priamo volané v HTML súbore webovej stránky. Kód bol spúštaný enginom webového prehliadača. Node.js umožňuje použitie JavaScriptu na strane servera a kód sa spúšťa predtým ako je dynamický obsah stránky poslaný do webového prehliadača používateľa. Vďaka tejto výhode sa z Node.js stal jeden zo základných elementov paradigmy "Javacript everywhere" ktorý pri vývoji webových aplikácii umožňuje zjednotenie programovacích jazykov namiesto spoliehania sa na iné programovacie jazyky nutne pre serverovú časť aplikácie.

\subsection{JavaScriptová knižnica Less.js}
Less je CSS pre=procesor, čo znamená, že rozširuje jazyk CSS, pridáva funkcionalitu ktorá umožňuje definovanie premenných, funkcií a množstvo ďalších techník. Vďaka tomu je možné vytvoriť kaskádové štýly, ktoré sú prehľadné, rozšíriteľné s možnosťou jednoduchej zmeny veľkého množstva faktorov ovplyvňujúcich vzhľad aplikácie. Zdrojové súbory v jazyku Less sú kompilované pomocou viacerých možných spôsobov. Napríklad priamo v prehliadači používateľa, v prostredí Node, PHP, .Net a ďalších. Výsledkom kompilácie je vytvorenie súboru s kaskádovými štýlmi vo formáte css.
 
\subsection{JavaScriptová knižnica Fabric.js}
Fabric.js je výkonná a jednoduchá JavaScriptová knižnica na prácu s grafickou plochou canvas. Poskytuje interaktívny model objektov na prácu s canvas elementami. Obsahuje taktiež parser na prácu s SVG grafickými súbormi, vďaka čomu je možné renderovať obsah SVG súboru do grafickej plochy canvas. Canvas síce umožňuje vytváranie objektov jednoduchých tvarov pomocou svojich natívnych funkcií, avšak následná manipulácia s týmito objektami je veľmi ťažkopádna. Fabric.js preto vytvoril sadu príkazov na jednoduchú manipuláciu s grafickými objektami. Stará sa o renderovanie a udržiavanie aktuálneho stavu plochy. 

\subsection{JavaScriptový framework Angular.js}
AngularJS je JavaScriptová voľne šíriteľná knižnica (framework), ktorá je zastrešovaná, vyvíjaná a udržiavaná prevažne zamestnancami spoločnosti Google. 

Princíp fungovania knižnice spočíva v tom, že po načítaní HTML stránky, do ktorej sú vložené vlastné značky atribútov, ktoré angular interpretuje ako smerníky naviazané na model. Ten je reprezentovaný štandardnými premennými jazyka JavaScript. Hodnoty týchto premenných je možné nastaviť manuálne v rámci kódu aplikácie, alebo získať zo statických alebo dynamických JSON súborov. Framework prispôsobuje a rozširuje tradičné HTML tak, aby prezentoval dynamický obsah prostredníctvom obojsmernej väzby dát, ktorá umožňuje automatickú synchronizáciu modelov a šablón.

\subsection{JavaScriptová knižnica Socket.io}
Socket.IO je JavaScriptová knižnica vytvorená pre realtime webové aplikácie. Socket.IO umožňuje realtime obojstrannú komunikáciu medzi webovým klientom a servermi. Socket.IO má dve časti:
- klientská strana ktorá beží vo webovom prehliadači a serverovú časť pre node.js.Obe časti majú takmer identické API. Podobne ako node.js je socket.io "event-driven", teda funcionalita je vyvolávaná pomocou volaní definovaných funckií ktoré sa volajú pri určitých akciach používateľa alebo systému.

Socket.io primárne používa WebSocket protokol. Aj ked by to mohol byť použítý len ako wrapper pre WebSocket, socket.io ponúka veľa dodatočných funckií vrátanie vysielania na viacero socketov, ukladanie dát pre každého kilenta a asynchrónne vstupno/výstupné operácie.
Socket.io v mojej implementácií mediawiki je inštalovaný pomocou balíčkového manažéra npm.

\subsection{WebSocket}
WebSocket je počítačový komunikačný protokol poskutujúci plny duplexný (obojsmerný)  komunkačný kanál cez jedno TCP pripojenie.
Protokol WebSocket bol štandarizovaný komisiou IEFT ako RFC 6455 v roku 2011 a WebSocket API pre webové IDL bolo štandarizované konsorciom W3C.

WebSocket bol navrhnutý tak, aby mohol byť spúštaný vo webových prehliadačoch a na webových serveroch, kde môže byť použitá ľubovolná  klientská a serverová aplikácia. WebSocket je nezávislý protokol založený na základe TCP. Jeho jediný vzťah ku HTTP je, že jeho handshake (naviazanie spojenia) je interpretované HTTP servermi ako požiadavok na aktualizáciu. Protokol WebSocket umožnuje interakciu medzi prehliadačom a webovým serverom s nižšou réžiou, uľahčuje real-time prenos dát (prenos dát v reálnom čase) zo servera na server. To je možne preto, že poskytuje štandarizovaný spôsob pre odosielanie obsahu zo serveru do prehliadača bez toho aby to bolo na požiadavku klienta (napríklad ako v AJAX volaní) a umožňuje predanie správ tam a naspäť pokiaľ je udržanie otvorené spojenie. Týmto spôsobom môže prebiehať obojsmerná komunikácia medzi webovým prehliadačom a serverom. Táto komunikácia prebieha klasicky cez TCP port s číslom 80 alebo s portom 443 v prípade ak sa jedná o šiforvané TLS spojenie. Toto je výhoda v prostrediach kde sú webové pripojenia k internetu blokované firewallom. Podobne obojsmernej komunikácie prehliadač-server bolo v minulosti dosiahnutej pomocou neštandartizovanými spôsobmi ako napríklad knižnica Comet.

Protokol WebSocket je v súčastnosti podporovaný vo väčšine webových prehliadačov. WebSocket ale taktiež vyžaduje podporu na strane serverom a webových aplikácií.

\section{Rozšírenia MediaWiki}

Systém MediaWiki ja navrhnutý spôsobom, ktorý umožňuje pridávať funkcionalitu pomocou rozšírení (angl. Extensions). Rozšírenia sú časti zdrojového kódu, ktorý môže ovplyvniť vizuálne prvky stránky alebo funkčné vlastnosti systému. Vzhľadom na kategóriu rozšírenia sa dajú využiť napríklad na:

\begin{itemize}
	\item rozšírenie wikitext syntaxe o vlastné prvky (Parser Extensions Category)
	\item pridanie schopnosti reportovania informácií (Special page Extensions Category)
	\item zmena vzhľadu MediaWiki (User interface Extensions)
	\item zvýšenie zabezpečenia pomocou vlastných autentifikačných mechanizmov (Authentication and Authorization Extensions Category)
\end{list}

Zoznam kategorizovaných rozšírení je možné nájsť na stránkach tvorcov MediaWiki systému. Množstvo rozšírení je vytvorených a udržiavaných samotnými autormi systému, avšak vzhľadom na to, že zdrojové kódy sú voľne dostupné, veľa ich pochádza od vývojárov tretej strany. 

Rozšírenia môžu byť do systému inštalované iba s administrátorským prístupom do súborového systému na serveri. Inštalácia pozostáva zo skopírovania zdrojových súborov do adresára \inlinecode{PHP}{\$IP\/extensions\/nazov_rozsirenia\/}. Následne ho je možné inicializovať v koreňovom adresári MediaWiki inštalácie, v súbore \inlinecode{PHP}{LocalSettings.php}, pomocou globálnej funkcie\\
 \inlinecode{PHP}{wfLoadExtension(\'nazov_rozsirenia\')}.

Pre správne fungovanie rozšírenia je potrebné zabezpečiť niekoľko kľúčových vlastností, pozostávajúcich z týchto krokov:
\begin{itemize}
	\item Zaregistrovať akýkoľvek media handler, parsovaciu funckiu, špeciálne stránky, vlastné XML značky a premenné použité vašim rozšírením.
	\item Definovať a zvalidovať každú konfiguračnú premennú ktorá je zadefinovaná vo vašom rozšírení.
	\item Pripraviť triedy použité vo vašom rozšírení pre autonačítanie.
	\item Rozhodnúť ktoré časti inštalačného nastavenia majú byť vykonané okamžite a ktorá majú čakať pokiaľ sa inicializuje jadro MediaWiki.
	\item Definovať dodatočné hooky potrebné pre rozšírenie.
	\item Vytvoriť / skontroluje všetky nové databázové tabuľky nutné pre rozšírenie.
	\item Nastaví lokalizáciu vášho rozšírenia.
\end{itemize}


\subsection{ResourceLoader}

ResoureLoader je tzv."delivery system" pre optimalizáciu načítavania a manažovania modulov.

Každá stránka postavená na MediaWiki obsahuje stovky kilobajtov Javascriptu. Vo veľa prípadoch sa niektoré časti tohto kódu nevyužijú z dôsledku používania nepodporovaného prehliadača alebo sa kód na danej stránke nepoužíva. V týchto prípadoch je čas načítania zbytočne dlhší ako je potrebné.

ResourceLoader tento problém rieši načítavaním zdrojových kódov "resources" na vyžiadanie a len na prehliadačoch ktoré ho podporujú. Tento proces pozostáva z nasledujúcich bodov:

\begin{enumerate}
	\item Minifikácia a zjednotenie
	\begin{itemize}
		\item Redukuje veľkosť kódu a čas na jeho parsovanie alebo sťahovanie. 
		\item Javascript a CSS súbory sú načítavané v jednej špeciálne formátovanej ""ResourceLoader Implement" serverovej odpovede.
	\end{itemize}
	\item Načítavanie v dávkách
	\begin{itemize}
		\item Redukuje počet serverových volaní.
		\item Serverová odpoveď pre načítanie podporuje načítanie viacerých modulov na jedno volanie, ktoré už obsahuje minifikované a spojené súbory.
	\end{itemize}
	\item Data URI vkladanie
	\begin{itemize}
		\item Ešte viac redukuje počet požiadavok na server, čas odozvy a využitie pásma.
		\item Obrázky referencované v CSS môžu byť vložené ako data URI linky.
	\end{itemize}
\end{enumerate}


\section{Cieľ práce}
\begin{itemize}
	\item Navrhnúť prostredie grafického editora
	\item Implementovať grafický editor do prostredia MediaWiki pomocou rozšírenia
	\item Navrhnúť a implementovať spôsob ukladania verzií výstupných súborov grafického editora
	\item Navrhnúť a implementovat komunikačný protokol na synchronizáciu grafického editora pre viacero nezávislých používateľov
	\item Integrovať rozšírenie s webovou stránkou fakulty http://wiki.matfyz.sk
\end{itemize}




\documentclass[12pt, a4paper, oneside]{book}
\usepackage[hidelinks]{hyperref}
\usepackage[slovak]{babel}
\usepackage{epsfig}
\usepackage{epstopdf}
\usepackage[chapter]{algorithm}
\usepackage{algorithmic}
\usepackage{listings}
\usepackage{amsmath}
\usepackage{amssymb}
\usepackage{graphicx}
\usepackage{multirow}
\usepackage{color}
\usepackage{url}
\usepackage[utf8]{inputenc}
\usepackage[T1]{fontenc}
\usepackage{setspace}
\usepackage{tabularx}
\usepackage{textcomp}
\usepackage{caption}
\usepackage{natbib}
\usepackage{nomencl}
\usepackage{graphicx}
\usepackage{subcaption}
\usepackage{placeins}
\usepackage[avantgarde]{quotchap}
\usepackage{titlesec}
\usepackage{array}
\usepackage[table,xcdraw]{xcolor}
\usepackage{enumitem}

\setstretch{1.5}
%\renewcommand\baselinestretch{1.5} % riadkovanie jeden a pol

\input listings.tex

\titleformat{\paragraph}
{\normalfont\normalsize\bfseries}{\theparagraph}{1em}{}
\titlespacing*{\paragraph}
{0pt}{3.25ex plus 1ex minus .2ex}{1.5ex plus .2ex}

% pekne pokope definujeme potrebne udaje
\newcommand\mftitle{Kolaboratívny grafický editor pre MediaWiki}
\newcommand\enmftitle{Collaborative graphics editor for MediaWiki}
\newcommand\mfthesistype{Diplomová práca}
\newcommand\enmfthesistype{Master thesis}
\newcommand\mfauthor{Bc. Martin Krasňan}
\newcommand\mfconsultant{Mgr. Ján Kľuka, PhD.}
\newcommand\mfadvisor{doc. RNDr. Zuzana Kubincová, PhD.}
\newcommand\mfplacedate{Bratislava, 2018}
\newcommand\mfuniversity{UNIVERZITA KOMENSKÉHO V BRATISLAVE}
\newcommand\mffaculty{FAKULTA MATEMATIKY, FYZIKY A INFORMATIKY}
\newcommand\mfpracovisko{Katedra aplikovanej informatiky}
\newcommand\enmfpracovisko{Department of Applied Informatics}


%referencie
\newcommand{\refimg}[1]{(Obrázok  \ref{#1})}
\newcommand{\reftab}[1]{(Tabuľka  \ref{#1})}
\newcommand{\reflst}[1]{(Zdrojový kód  \ref{#1})}

%\newcommand{\p}[1]{\paragraph{#1}\mbox{}\\}


\newcommand{\imageHeight}{150px}


\ifx\pdfoutput\undefined\relax\else\pdfinfo{ /Title (\mftitle) /Author (\mfauthor) /Creator (PDFLaTeX) } \fi

\begin{document}

\frontmatter

\setcounter{secnumdepth}{2}

\thispagestyle{empty}

\noindent
\begin{minipage}{\textwidth}
\begin{center}
\textbf{\mfuniversity \\
\mffaculty}
\end{center}
\end{minipage}

\vfill
\begin{figure}[!hbt]
	\begin{center}
		\includegraphics{images/base/logo_fmph}
		\label{img:logo}
	\end{center}
\end{figure}
\begin{center}
	\begin{minipage}{0.8\textwidth}
		\centerline{\textbf{\Large\MakeUppercase{\mftitle}}}
		\smallskip
		\centerline{\mfthesistype}
	\end{minipage}
\end{center}
\vfill
2018 \hfill
\mfauthor
\eject 
% koniec obalu

\thispagestyle{empty}

\noindent
\begin{minipage}{\textwidth}
	\begin{center}
		\textbf{\mfuniversity \\
			\mffaculty}
	\end{center}
\end{minipage}


\vfill
\begin{figure}[!hbt]
\begin{center}
\includegraphics{images/base/logo_fmph_dark}
\label{img:logo_dark}
\end{center}
\end{figure}
\begin{center}
\begin{minipage}{0.8\textwidth}
\centerline{\textbf{\Large\MakeUppercase{\mftitle}}}
\smallskip
\centerline{\mfthesistype}
\end{minipage}
\end{center}
\vfill
\begin{tabular}{l l}
%Registration number: & 40a99bd8-3cb6-4534-9330-c7fd9b5e5ca4 \\
Študijný program: & Aplikovaná informatika\\
Študijný odbor: & 2511 Aplikovaná informatika\\
Školiace pracovisko: & Katedra aplikovanej informatiky\\
Školiteľ: & \mfadvisor\\
Konzultant: & \mfconsultant
\end{tabular}
\vfill
\noindent
\mfplacedate \hfill
\mfauthor
\eject 
% koniec titulneho listu

%\thispagestyle{empty}
%\includegraphics[width=\textwidth]{images/base/zadanie}
%\vfill
%\eject
% koniec zadania
\setcounter{page}{2}
\thispagestyle{empty}


\begin{figure}[H]
\begin{center}
\makebox[\textwidth]{\includegraphics[width=\paperwidth]{images/base/zadaniedp}}
\label{img:zadanie}
\end{center}
\end{figure}

{~}\vspace{12cm}

\noindent
\begin{minipage}{0.25\textwidth}~\end{minipage}
\begin{minipage}{0.75\textwidth}
Čestne vyhlasujem, že som túto diplomovú prácu vypracoval samostatne pod vedením doc. RNDr. Zuzany Kubincovej, PhD., s použitím zdrojov uvedených v zozname použitej literatúry.
\newline \newline
\end{minipage}
\vfill
~ \hfill {\hbox to 6cm{\dotfill}} \\
\mfplacedate \hfill \mfauthor
\vfill\eject 
% koniec prehlasenia

\chapter*{Poďakovanie}\label{chap:thank_you}
Ďakujem doc. RNDr. Zuzane Kubincovej PhD. za čas a trpezlivosť, ktorú mi venovala pri konzultáciách, za cenné rady a profesionálne vedenie pri vypracovaní diplomovej práce. Moja veľká vďaka patrí taktiež Mgr. Jánovi Kľukovi, PhD. za ochotu a pomoc pri riešení problémov spojených s touto prácou. V neposlednom rade sa chem poďakovať mojim rodičom, starým rodičom a sestre za ich podporu. 
\vfill\eject 
% koniec podakovania

\def\spaceafterpar{1em}
\setlength{\parskip}{\spaceafterpar}
\setlength\parindent{0pt}

\section*{Abstrakt}\label{chap:abstract_sk}
\MakeUppercase{Krasňan}, Martin: \textit{\mftitle} [\mfthesistype]. - Univerzita Komenského v Bratislave. Fakulta matematiky, fyziky a informatiky; \mfpracovisko. - Školiteľ: \mfadvisor. Bratislava: \mfplacedate. 85 strán.
%TODO: pocet stran

Pri súčasnom rozvoji používania internetu a webových aplikácií je vítanou možnosťou kolaboratívne vytváranie informácií, ktoré sú na internete vyhľadávané a udržiavané. Jedným z používaných nástrojov umožňujúcich tento proces je systém MediaWiki.
Cieľom diplomovej práce bolo navrhnúť a implementovať grafický editor pre systém MediaWiki, umožňujúci kolaboratívne kreslenie a úpravu obrázkov. Vytvorený editor bolo potrebné integrovať do stránky wiki.matfyz.sk. Nami implementované riešenie pozostáva z dvoch častí. Prvú časť tvorí implementácia grafického editora vektorových obrázkov. Integrácia do systému MediaWiki bola realizovaná formou rozšírenia. V druhej časti sme vytvorili komunikačný protokol a synchronizačný server, slúžiaci na výmenu informácií o obsahu grafickej plochy editorov spolupracujúcich používateľov. 

~\\
\textbf{Kľúčové slová:} kolaboratívny grafický editor, webová aplikácia, rozšírenie MediaWiki
\vfill\eject 

\section*{Abstract}\label{chap:abstract_en}
\MakeUppercase{Krasňan}, Martin: \textit{\enmftitle} [\enmfthesistype]. - Comenius University in Bratislava. Faculty of Mathematics, Physics and Informatics; \enmfpracovisko. - Supervisor: \mfadvisor. \mfplacedate. 85 pages.
%TODO: pocet stran

At a current rate of the Internet and web application expansion, it is a welcome opportunity to collaboratively cerate and share information that is searched and maintained on the Internet. One of the tools used to enable this process is MediaWiki. The aim of the diploma thesis was to design and implement a graphic editor for MediaWiki, enabling collaborative drawing and editing of images. The created editor needed to be integrated into wiki.matfyz.sk. The solution implemented by us consists of two parts. The first part consists of implementing a graphic editor of vector images. Integration into MediaWiki has been implemented as an extension. In the second part, we have created a communication protocol and a synchronization server to exchange information about the content of the collaborative user's graphics area.

~\\
\textbf{Keywords:} collaborative graphics editor, web application, MediaWiki extension
\vfill\eject 
% koniec abstraktov


\setlength{\parskip}{0em}
% treba este prejst dokument ci je kod spravne formatovany
\tableofcontents

%zrusenie indentu pre paragraph a nastavenie medzery nad novym riadkom
\setlength{\parskip}{\spaceafterpar}


\mainmatter


\input intro.tex
\input issuesoverview.tex
\input previoussolutions.tex
\input proposal.tex
\input implementation.tex
\input results.tex
\input conclusion.tex

\backmatter
\def\spaceafterpar{0em}
\setlength{\parskip}{\spaceafterpar}
\listoffigures

\listoftables



\nocite{*}
\bibliographystyle{alpha}
\bibliography{references}

\chapter*{Príloha A - obsah elektronického média}
\addcontentsline{toc}{chapter}{Príloha A - obsah elektronického média}
\begin{itemize}
	\item súbor s textom diplomovej práce vo formáte pdf,
	\item priečinok \textbf{obrázky}, v ktorom sú všetky obrázky použité v texte diplomovej práce,
	\item priečinok \textbf{zdrojové súbory rozšírenie}, obsahuje implementáciu rozšírenia pre systém MediaWiki,
		\item priečinok \textbf{zdrojové súbory server}, obsahuje implementáciu synchronizačného servera
\end{itemize}


\end{document}

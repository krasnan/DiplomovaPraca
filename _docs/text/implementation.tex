\chapter{Implementácia}\label{chap:implementation}
V tejto kapitole sa zameriame na riešenie popísanej problematiky, implementácii algoritmov a funkcií potrebných na správnu funkčnosť a dosiahnutie cieľov zadania diplomovej práce. Implementácia pozostáva z dvoch hlavných častí. Vytvorenie synchronizačného servera a klientskej aplikácie grafického editora formou rozšírenia systému MediaWiki. Popíšeme si ich dátové modely a ich funkcionalitu. 

Pri implementácií využívame verziovací systém git. Obe hlavné časti sú umiestnené vo vlastných repozitároch na serveri GitHub. 

\section{Synchronizačný server}
Synchronizačný server pozostáva z JavaScriptovej aplikácie naprogramovanej v jazyku JavaScript. Aplikácia je spúšťaná v runtime prostredí NodeJS servera, konkrétne vo verzii 7.6 a vyššej. Ten je nainštalovaný formou kontajnerového systému docker. Aplikácia pracuje pomocou inštancie HTTP servera, využívajúceho port s číslom \code{8080} a lokálnu adresou servera \code{0.0.0.0}. 

Knižnice servera sú inštalované pomocou balíčkovacieho manažéra NPM (\textbf{N}ode \textbf{P}ackage \textbf{M}anager). Ich potrebné verzie sú zapísané v súbore \code{package.json}. Nachádzajú sa tu taktiež konfiguračné premenné prostredia:

\begin{itemize}
	\item \code{api_endpoint} - adresa API koncového bodu systému MediaWiki, v ktorom je nainštalované naše rozšírenie. V našom prípade má hodnotu \code{\"https://wiki.matfyz.sk/api.php\"}.
	
	\item \code{api_token} - tajný token slúžiaci na overenie pripájaných klintov. Jeho hodnota sa musí zhodovať s hodnotou na strane rozšírenia.
\end{itemize}

\subsection{Dátový model}
Pre implementáciu serverovej časti aplikácie využívame dátovú štruktúru JavaScriptových objektov.

\subsubsection{User}
Trieda \code{User} reprezentuje používateľa pripojeného na synchronizačný server. Jej vlastnosti určujú triedne premenné a metódy opísané v tabuľkách \ref{tab:server-prop-user} a \ref{tab:server-func-user}.

\begin{table}
	\begin{tabular}{ | m{3cm} | m{3cm}| m{6.5cm} | } \hline
		\textbf{Názov} & \textbf{Typ} & \textbf{Popis} \\ \hline
		
		name & string & Meno používateľa prijaté v dopyte pri pripojení klienta na server \\\hline
		color & string & Automaticky generovaná farba pomocou funkcie \code{getRandomColor()} v HEX formáte {\code{\"#ffff00\"}} \\\hline
		verified & boolean & Príznak, či bol korektne overený bezpečnostný token \\\hline
		token & string & Privátna premenná s hodnotou bezpečnostného tokenu \\\hline
		
		\hline
	\end{tabular}
	\caption{Zoznam triednych premenných objektu User}
	\label{tab:server-prop-user}
\end{table}


\begin{table}
	\begin{tabular}{ | m{4cm} | m{8.5cm} | } \hline
		\textbf{Názov} & \textbf{Popis} \\ \hline
		
		setToken(token) & Nastavenie bezpečnostného tokenu  \\\hline
		getToken() & Vráti hodnotu bezpečnostného tokenu \\\hline
		verifyUser() & Overí hodnotu bezpečnostného tokenu a nastaví triednu premennú \code{verified} \\\hline
		
		\hline
	\end{tabular}
	\caption{Zoznam metód objektu User}
	\label{tab:server-func-user}
\end{table}


\subsubsection{Message}
Objekt \code{Message} reprezentuje správu odoslané používateľmi pomocou chatu v editore. Popis vnútornej reprezentácie vlastností objektu môžeme vidieť v tabuľke \ref{tab:server-prop-message}

\begin{table}
	\begin{tabular}{ | m{3cm} | m{3cm}| m{6.5cm} | } \hline
		\textbf{Názov} & \textbf{Typ} & \textbf{Popis} \\ \hline
		
		from & User & Odosielateľ správy \\\hline
		to & User & Prijímateľ správy \\\hline
		text & string & Textový obsah správy \\\hline
		type & string & Typ správy nastavovaný ak ide o systémovú správu \\\hline
		time & string & Dátum a čas odoslania správy \\	\hline
	
		\hline
	\end{tabular}
	\caption{Zoznam triednych premenných objektu Message}
	\label{tab:server-prop-message}
\end{table}



\subsubsection{Room}
Trieda \code{Room} slúži na udržiavanie stavu miestnosti pripojených používateľov, spracovanie požiadaviek odosielaných zo strany editora, udržiavanie štruktúry objektov, odoslaných správ a funkcionality na načítanie informácií o súbore. Popis jej vnútornej štruktúry a poskytovanej funkcionality môžeme vidieť v tabuľkách \ref{tab:server-prop-room} a \ref{tab:server-func-room}.

\begin{table}
	\begin{tabular}{ | m{3cm} | m{3cm}| m{6.5cm} | } \hline
		\textbf{Názov} & \textbf{Typ} & \textbf{Popis} \\ \hline
	
		users & array<User> & Zoznam používateľov \\\hline
		messages & array<Message> & Zoznam správ \\\hline
		objects & array<object> & Zoznam objektov grafickej plochy \\\hline
		canvas & object & Dynamický objekt s nastaveniami grafickej plochy \\\hline
		format & string & Formát editovaného súboru (jpg, png, svg)\\\hline
		loaded & boolean & Príznak či je miestnosť plne vyinicializovaná \\\hline
		file & string & Názov editovaného súboru \\\hline
	
		\hline
	\end{tabular}
	\caption{Zoznam triednych premenných objektu Room}
	\label{tab:server-prop-room}
\end{table}


\begin{table}
	\begin{tabular}{ | m{6cm} | m{6.5cm} | } \hline
		\textbf{Názov} & \textbf{Popis} \\ \hline
		
		loadFromWiki() & Načíťanie obsahu editora pri inicializácii miestnosti. Načítanie prebieha pomocou HTTP dopytu na koncový bod API systému MediaWiki, v ktorom je rozšírenie nainštalované. \\\hline
		createUser(user, socket) & Pripojí používateľa do zvolenej miestnosti na základe editovaného súboru. \\\hline
		removeUser(user, socket) & Odstáni používateľa zo zoznamu používateľov a odpojí ho z miestnosti.  \\\hline
		createMessage(text, from, to, type) & Spracováva požiadavku odoslania novej správy pomocou chatu. \\\hline
		modifyCanvas(properties, socket) & Spracováva požiadavku na zmenu grafickej plochy editora. \\\hline
		createObject(object, socket) & Spracovanie požiadavky na vytvorenie objektu grafického editora. \\\hline
		modifyObject(object, socket) & Spracovanie požiadavky na zmenu objektu grafického editora. \\\hline
		removeObject(id, socket) & Spracovanie požiadavky na odstránenie objektu grafického editora. \\\hline
		setSelectable(id, selectable, user, socket) & Spracovanie požiadavky na zmenu uzamknutia objektu grafickej plochy. \\\hline
		deselectAll() & Metóda nastaví všetky objekty grafickej plochy na neuzamknuté. \\\hline
		
		\hline
	\end{tabular}
	\caption{Zoznam metód objektu Room}
	\label{tab:server-func-room}
\end{table}


\subsubsection{RoomManager}
Trieda menežéra miestností \code{RoomManager} slúži na spravovanie existujúcich miestností, ich dynamické vytváranie a vymazávanie. Obsahuje triednu premennú \code{rooms}. Je to premenná typu asociatívne pole, obsahujúca objekty typu \code{Room}. Poskytuje funkcie na prácu s miestnosťami popísané v tabuľke \ref{tab:server-func-roommanager}.

\begin{table}
	\begin{tabular}{ | m{4cm} | m{8.5cm} | } \hline
		\textbf{Názov} & \textbf{Popis} \\ \hline
		
		isEmpty() & Metóda vracajúca true/false hodnotu v závislosti od toho, či je v danej miestnosti pripojený nejaký používateľ. \\\hline
		createRoom(name) & Metóda overí či existuje miestnosť s daným názvom a v prípade že neexistuje, vytvorí ju. Návratovou hodnotou je objekt typu \code{Room} zodpovedajúci danému názvu miestnosti. \\\hline
		getRoom(name) & Metóda vráti objekt typu \code{Room} v prípade že existuje miestnosť s daným názvom v zozname miestností. \\\hline
		removeRoom(name) & Metóda vymaže miestnosť so zadaným názvom zo zoznamu miestností. \\\hline
		
		\hline
	\end{tabular}
	\caption{Zoznam metód objektu RoomManager}
	\label{tab:server-func-roommanager}
\end{table}
\FloatBarrier

\subsection{Pripojenie klienta}

Pri vytvorení inštancie Socket.IO objektu s názvom \code{io}, vkladáme do jeho konštruktora inštanciu vytvoreného HTTP servera. Po jej vytvorení je server pripravený prijímať správy s udalosťami takzvané \textit{socket}-y. Príchod pripájacej správy s popisom udalosti  \code{\'connection\'} je odchytený event-listenerom  \code{io.on(\'connection\', function(socket)\{...\})}, v ktorom sa volá funkcia spracujúca túto udalosť. Informácie o pripájanom klientovi je do nej poslaná parametrom socket. Pomocou tohoto parametra sú následne server odchytáva správy ďalšie správy odosielané zo strany klienta. Parameter v sebe taktiež nesie objekt \code{socket.handshake}. V tomto objekte sú uložené informácie o nadviazanom spojení, medzi ktorými je aj premenná query. V tejto premennej sú uložené informácie o názve editovaného súboru, bezpečnostný token a meno pripájaného používateľa.



\begin{figure}[h]
	\centerline{\includegraphics[width=1\textwidth]{images/diagrams/sequence-diagram-server-connect}}
	\caption[Sekvenčný diagram pripojenia používateľa]{Sekvenčný diagram úspešného pripojenia používateľa na synchronizačný server}
	\label{obr:sequence-diagram-server-connect}
\end{figure}
\FloatBarrier

\subsubsection{Vytvorenie používateľa}

\subsubsection{Overenie používateľa}

\subsubsection{Inicializácia miestnosti}


\subsection{Spracovanie operácií grafického editora}

\subsubsection{Zmeny grafickej plochy}

\subsubsection{Vytvorenie objektu}

\subsubsection{Zmena uzamknutia objektu}

\subsubsection{Modifikácia objektu}

\subsubsection{Odstránenie objektu}

\subsubsection{Textová správa}


\subsection{Odpojenie klienta}

\subsubsection{Odstránenie miestnosti}
